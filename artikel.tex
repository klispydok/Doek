\documentclass[11pt,a4paper]{article}

\date{01.11.2014} 

\author{EVa \and dario}

\title{\"UBERSCHRIFT} 

\begin{document} 

\maketitle 

\renewcommand{\abstractname}{kleine zusammenfassung}
\begin{abstract} 

inhaltsangabe folgt 

\end{abstract} 

\renewcommand{\contentsname}{inhaltsangabe}
\tableofcontents 

\section{Abschnitt} 

Text 

\section{Ich war hier}
Gruess, Oliver.


\section{Tabelle} 


\begin{tabular}{ |l| c |c |r| r|}
\hline
  1 & 2 & 3 & c & c\\
  \hline
  4 & 5 & 6 & c & noline here\\
 
  7 & 8 & 9 & c & c\\
  \hline
\end{tabular} 

\raggedbottom
\pagebreak
\section{Formel}
ein paar formeln

\subsection{pythagoras}
a^2+b^2=c^2


\subsubsection{bei nichtrechtwinkligen dreiecken}
\raggedright
der cosinus satz c^2=a^2+b^2-2ab\cos\gamma \end{textit} \linebreak

resp f\"ur a,b \linebreak

a^2=b^2+c^2-2bc\cos\alpha \end{textit} \linebreak

b^2=a^2+c^2-2ac\cos\beta \end{textit} \linebreak


\subsection{eulersche zaal}
e= \lim_{x \to \infty}(1+\frac{1}{n})^n \linebreak \end{textit} \linebreak \end{textit}
e=\sum_{k=1}^{\infty} (\frac{1}{k!})

\section{Ich war hier}
Nathalia Münch

\end{document}
